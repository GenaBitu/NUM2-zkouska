\documentclass[a4paper]{article}

\usepackage{polyglossia}
\setdefaultlanguage{czech}
\usepackage[intlimits]{amsmath}

\begin{document}

\section*{Zkouškový příklad NUM2, Marek Dědič}

\subsection*{Zadání}
Řešte metodou střelby okrajovou úlohu
\begin{subequations}\label{input}
	\begin{align}
		-y'' + 16y = 8 \qquad x \in \left( 0, \frac{\pi}{2} \right) \label{problem} \\
		y \left( 0 \right) = \alpha \qquad y \left( \frac{\pi}{2} \right) = \beta \label{boundary}
	\end{align}
\end{subequations}

\subsection*{Úpravy}
Rovnice \eqref{problem} je nehomogenní lineární diferenciální rovnice s konstantními koeficienty, jejíž charakteristická rovnice má tvar
\begin{equation}\label{chareq1}
	- \lambda^2 + 16 = 0
\end{equation}
Tu lze upravit do tvaru
\begin{equation}\label{chareq2}
	\left( \lambda + 4 \right) \left( \lambda - 4 \right) = 0
\end{equation}
Tedy fundamentální systém řešení \eqref{problem} je \( \left\{ e^{ 4x }, e^{ -4x } \right\} \) a homogenní řešení má tvar
\begin{equation}\label{homogeneous}
	y_H = C_1 e^{ 4x } + C_2 e^{ -4x }
\end{equation}
Neboť pravá strana \eqref{problem} je konstantní, partikulární řešení lze snadno určit jako
\begin{equation}\label{particular}
	y_P = \frac{1}{2}
\end{equation}
A tedy obecné řešení \eqref{problem} bude tvaru
\begin{equation}\label{solution}
	y = C_1 e^{ 4x } + C_2 e^{ -4x } + \frac{1}{2}
\end{equation}
Pro určení koeficientů \( C_1 \) a \( C_2 \) využijeme tří různých přístupů.

\subsection*{Analytické řešení}
Tvar \eqref{solution} můžeme dosadit přímo do okrajových podmínek \eqref{boundary} a tím dostaneme soustavu
\begin{align}
	\begin{split}
		\alpha &= C_1 + C_2 + \frac{1}{2} \\
		\beta &= C_1 e^{ 2 \pi } + C_2 e^{ -2 \pi } + \frac{1}{2}
	\end{split}
\end{align}
Kterou lze vyřešit, čímž dostanu analytické řešení \eqref{input} ve tvaru
\begin{align}
	\begin{split}
		C_1 &= \frac{\left( 2 \beta - 1 \right) + e^{ -2 \pi } \left( 1 - 2 \alpha \right)}{4 sinh \left( 2 \pi \right)} \\
		C_2 &= \alpha - \frac{1}{2} - C_1 \\
		y &= C_1 e^{ 4x } + C_2 e^{ -4x } + \frac{1}{2}
	\end{split}
\end{align}

\subsection*{Metoda střelby}
Okrajovou úlohu \eqref{input} nahradíme počáteční úlohou tvaru
\begin{subequations}\label{shoot}
	\begin{align}
		-y'' + 16y = 8 \qquad x \in \left( 0, \frac{\pi}{2} \right) \\
		y \left( 0 \right) = \alpha \qquad y' \left( 0 \right) = \gamma
	\end{align}
\end{subequations}

\end{document}
