\documentclass[a4paper]{article}

\usepackage{polyglossia}
\setdefaultlanguage{czech}
\usepackage[intlimits]{amsmath}
\usepackage[top=1in, bottom=1.25in, left=1.25in, right=1.25in]{geometry}

\begin{document}

\section*{Zkouškový příklad NUM2, Marek Dědič}

\subsection*{Zadání}
Řešte metodou střelby okrajovou úlohu
\begin{subequations}\label{input}
	\begin{align}
		-y'' + 16y = 8 \qquad x \in \left( 0, \frac{\pi}{2} \right) \label{problem} \\
		y \left( 0 \right) = \alpha \qquad y \left( \frac{\pi}{2} \right) = \beta \label{boundary}
	\end{align}
\end{subequations}

\subsection*{Úpravy}
Rovnice \eqref{problem} je nehomogenní lineární diferenciální rovnice s konstantními koeficienty. Její charakteristická rovnice má tvar
\begin{equation}\label{chareq1}
	- \lambda^2 + 16 = 0
\end{equation}
Charakteristickou rovnici lze upravit do tvaru
\begin{equation}\label{chareq2}
	\left( \lambda + 4 \right) \left( \lambda - 4 \right) = 0
\end{equation}
Tedy fundamentální systém řešení \eqref{problem} je \( \left\{ e^{ 4x }, e^{ -4x } \right\} \) a homogenní řešení má tvar
\begin{equation}\label{homogeneous}
	y_H = C_1 e^{ 4x } + C_2 e^{ -4x }
\end{equation}
Neboť pravá strana \eqref{problem} je konstantní, partikulární řešení lze snadno určit jako
\begin{equation}\label{particular}
	y_P = \frac{1}{2}
\end{equation}
A tedy obecné řešení \eqref{problem} bude tvaru
\begin{equation}\label{solution}
	y = C_1 e^{ 4x } + C_2 e^{ -4x } + \frac{1}{2}
\end{equation}
Pro určení koeficientů \( C_1 \) a \( C_2 \) využijeme tří různých přístupů.

\subsection*{Analytické řešení}
Tvar \eqref{solution} můžeme dosadit přímo do okrajových podmínek \eqref{boundary} a tím dostaneme soustavu
\begin{align}
	\begin{split}
		\alpha &= C_1 + C_2 + \frac{1}{2} \\
		\beta &= C_1 e^{ 2 \pi } + C_2 e^{ -2 \pi } + \frac{1}{2}
	\end{split}
\end{align}
kterou lze vyřešit, čímž dostaneme analytické řešení \eqref{input} ve tvaru
\begin{align}
	\begin{split}
		C_1 &= \frac{\left( 2 \beta - 1 \right) + e^{ -2 \pi } \left( 1 - 2 \alpha \right)}{4 sinh \left( 2 \pi \right)} \\
		C_2 &= \alpha - \frac{1}{2} - C_1 \\
		y &= C_1 e^{ 4x } + C_2 e^{ -4x } + \frac{1}{2}
	\end{split}
\end{align}

\subsection*{Metoda střelby}
Okrajovou úlohu \eqref{input} nahradíme počáteční úlohou tvaru
\begin{subequations}\label{shoot}
	\begin{align}
		-y'' + 16y = 8 \qquad x \in \left( 0, \frac{\pi}{2} \right) \\
		y \left( 0 \right) = \alpha \qquad y' \left( 0 \right) = \gamma \label{shootBoundary}
	\end{align}
\end{subequations}
Kde zavádíme proměnnou \( \gamma \) a tedy \( y = y \left( x; \gamma \right) \). Pro nějaké \( \gamma^* \) platí
\begin{equation}
	y \left( \frac{\pi}{2}; \gamma^* \right) = \beta
\end{equation}
Definujeme funkci \( F \left( \gamma \right) \)
\begin{equation}
	F \left( \gamma \right) = y \left( \frac{\pi}{2}; \gamma \right) - \beta
\end{equation}
pro kterou platí 
\begin{equation}\label{Feq}
	F \left( \gamma^* \right) = 0
\end{equation}
Tvar \eqref{solution} dosadíme do počátečních podmínek \eqref{shootBoundary}, čímž získáme soustavu
\begin{align}
	\begin{split}
		\alpha &= C_1 + C_2 + \frac{1}{2} \\
		\gamma &= 4C_1 - 4C_2
	\end{split}
\end{align}
kterou lze upravit do tvaru
\begin{align}
	\begin{split}
		C_1 &= \frac{\alpha}{2} + \frac{\gamma}{8} - \frac{1}{4} \\
		C_2 &= \frac{\alpha}{2} - \frac{\gamma}{8} - \frac{1}{4}
	\end{split}
\end{align}
Dosadíme \eqref{solution} do \eqref{Feq}, čímž spolu s předcházejícími vztahy dostáváme úlohu
\begin{align}
	\begin{split}
		C_1 \left( \gamma \right) &= \frac{\alpha}{2} + \frac{\gamma}{8} - \frac{1}{4} \\
		C_2 \left( \gamma \right) &= \frac{\alpha}{2} - \frac{\gamma}{8} - \frac{1}{4} \\
		F \left( \gamma^* \right) &= C_1 \left( \gamma^* \right) e^{ 2 \pi } + C_2 \left( \gamma^* \right) e^{-2 \pi} + \frac{1}{2} - \beta = 0
	\end{split}
\end{align}
K hledání řešení \( \gamma^* \) můžeme využít dvou přístupů.

\subsubsection*{Metoda půlení intervalu}
Musíme nalézt počáteční hodnoty \( \gamma_L \) a \( \gamma_R \) takové, aby
\begin{equation}
	F \left( \gamma_L \right) F \left( \gamma_R \right) < 0
\end{equation}
Poté v každém kroku určíme
\begin{equation}
	\gamma_M = \frac{\gamma_L + \gamma_R}{2}
\end{equation}
a nahradíme
\begin{subequations}\label{halve}
	\begin{align}
		\gamma_L &= \gamma_M \qquad \text{pokud} \quad F \left( \gamma_L \right) F \left( \gamma_M \right) > 0 \\
		\gamma_R &= \gamma_M \qquad \text{jinak}
	\end{align}
\end{subequations}
Tím se nám interval \( \left( \gamma_L, \gamma_R \right) \) o polovinu zmenšil. Krok \eqref{halve} opakujeme, dokud nedosáhneme požadované přesnosti.

\subsubsection*{Newtonova metoda}
Vybereme náhodné \( \gamma^{ \left( 0 \right)} \) a v každém kroku nahradíme
\begin{equation}
	\gamma^{ \left( k + 1 \right) } = \gamma^{ \left( k \right) } - \frac{F \left( \gamma^{ \left( k \right) } \right) }{F' \left( \gamma^{ \left( k \right) } \right) }
\end{equation}

\end{document}
